\documentclass[11pt,a4paper]{article}
\usepackage{layout}
\usepackage{exscale, amsmath, amsfonts, amssymb, amsthm, amscd, fdsymbol, mdframed}

\usepackage{enumerate,textcomp}
%multicol}
\usepackage{subfigure}
\usepackage{graphicx}
%\usepackage{color}
\usepackage{float}

\usepackage{xspace}

%\usepackage[notcite,notref]{showkeys}
%\usepackage[notcite]{showkeys}
%\usepackage{showtags}
\usepackage{verbatim}
\parindent0pt
\parskip7pt
\textwidth14.5cm
\textheight=24cm
\addtolength{\oddsidemargin}{-7mm}
\addtolength{\evensidemargin}{-7mm}
\addtolength{\voffset}{-15mm}
\pagestyle{myheadings}
\renewcommand{\baselinestretch}{0.99mm}

\usepackage{hyperref}

%============================= MAKROS ===============================
\newcommand{\no}{\noindent}
\newcommand{\pfskip}{\vspace{3mm}}
\newcommand{\qqd}{\quad\quad}

%-------------------- General Math ----------------------
\newcommand{\Conv}{{\rm conv\,}}
\newcommand{\Area}{{\rm area\,}}
\newcommand{\Inter}{{\rm int\,}}
\newcommand{\dist}{{\rm dist}}
\newcommand{\N}{\mathbb{N}}
%\renewcommand{\P}{\mathbb{P}}
\newcommand{\E}{\mathbb{E}}
%\newcommand{\K}{\mathbb{K}}
\newcommand{\Q}{\mathbb{Q}}
\newcommand{\R}{\mathbb{R}}
\newcommand{\Z}{\mathbb{Z}}
\newcommand{\T}{\mathbb{T}}

%--------------------- Software ------------------------
\newcommand{\DetectorChecker}{\emph{DetectorChecker}\xspace}
\newcommand{\DetectorCheckerWebApp}{\emph{DetectorChecker webapp}\xspace}
\newcommand{\Rsoftware}{\emph{R}\xspace}

%--------------------- Probability ------------------------
\newcommand{\w}{\omega}
%\newcommand{\wt}{{\widetilde{\omega}}}
\newcommand{\W}{\Omega}
\newcommand{\MW}{M\times\W}
\newcommand{\dmP}{d \mu\hspace{-1mm}\otimes\hspace{-1mm}P}
\newcommand{\mP}{\mu\hspace{-0.5mm}\otimes\hspace{-0.5mm}P}
\newcommand{\A}{{\mathcal A}}
\newcommand{\B}{{\mathcal B}}
\newcommand{\C}{{\mathcal C}}
\newcommand{\F}{{\mathcal F}}
\newcommand{\G}{{\mathcal G}}
\newcommand{\I}{{\mathcal I}}
\newcommand{\J}{{\mathcal J}}
\renewcommand{\L}{{\mathcal L}}
\newcommand{\M}{{\mathcal M}}
\renewcommand{\P}{{\mathcal P}}
\newcommand{\Tail}{{\mathcal T}}
\newcommand{\eps}{\varepsilon}


%........................................
\newcommand{\Lc}{{\widehat{L}}}
\newcommand{\Lf}{{\widehat{L}}^\sharp}
\newcommand{\Lbm}{{\widehat{L}}^{\flat-}}

%\newcommand{\lowstepline}{-\hspace{-2mm}\bullet\hspace{-2.38mm}^{\mid}\hspace{1mm}}
\newcommand{\upstepline}{{\shortmid\hspace{-1.27mm}}^{\bullet\hspace{-0.7mm}-}}
\newcommand{\lowstep}{-\hspace{-1mm}\bullet\hspace{-1.15mm}^{\mid}}
\newcommand{\upstep}{{\shortmid\hspace{-1mm}}^{\bullet\hspace{-1mm}-}}

\newcommand{\partialc}{{\widehat{\partial}}}
%\newcommand{\Pa}{{\mathcal P}}
\newcommand{\len}{{\rm length}\,}
\newcommand{\area}{{\rm area}\,}
%\newcommand{\inter}{{\rm int}\,}

\newcommand{\nd}{\noindent}
\newcommand{\TTT}{\clubsuit}

%*********************** Theorem, Definition... *****************
%\newcommand{\Definition}[1]{\newtheorem{%***********************
%\newtheorem{satz}{Satz}[section]
\newtheorem{thm}{{\bf  Theorem}}[section]
\newtheorem{lemma}[thm]{{\bf  Lemma}}
\newtheorem{cor}[thm]{{\bf  Corollary}}
\newtheorem{prop}[thm]{{\bf  Proposition}}
\newtheorem{definition}[thm]{{\bf  Definition}}
\newtheorem{rem}[thm]{{\bf Remark}}
\newtheorem{ex}[thm]{{\bf  Example}}
\newcommand{\pf}{\noindent {\bf Proof. }}
\newcommand{\pfofthm}{\noindent {\bf Proof of the theorem. }}
\newcommand{\pfoflem}{\noindent {\bf Proof of the lemma. }}
\newcommand{\nn}{\nonumber}
\newcommand{\noind}{\noindent}

\renewcommand{\baselinestretch}{1.2}

%>>>>>>>>>>>>>>>>>>>>>>>>>>>>>>>>>>>>>>>>>>>>>>>>>>>>>>>>>>>>>>>>>>>
\begin{document}
%>>>>>>>>>>>>>>>>>>>>>>>>>>>>>>>>>>>>>>>>>>>>>>>>>>>>>>>>>>>>>>>>>>>

%TTTTTTTTTTTTTTTTTTTTTTTTTTTTTTTTTTTTTTTTTTTTTTTTTTTTTTTTT
% Title
%TTTTTTTTTTTTTTTTTTTTTTTTTTTTTTTTTTTTTTTTTTTTTTTTTTTTTTTTT

\begin{center}

{\LARGE\bf DetectorChecker Instruction Manual}
\footnote{This manual refers to version 1.0.8 of \DetectorChecker and version 1.0.8 of \DetectorCheckerWebApp.}

%{\Large For release version $\beta$}

\bigskip
{\large Julia A Brettschneider, Tomas Lazauskas, Oscar T Giles, Wilfrid S Kendall}

\end{center}


%\renewcommand{\thefootnote}{}
%\footnote{2000 Mathematics Subject Classification: ...}
%\vspace{-5mm}


%SSSSSSSSSSSSSSSSSSSSSSSSSSSSSSSSSSSSSSSSSSSSSSSSSSSSSSSSSSSSSSSSSSSS
\section{Introduction}\label{int}
%SSSSSSSSSSSSSSSSSSSSSSSSSSSSSSSSSSSSSSSSSSSSSSSSSSSSSSSSSSSSSSSSSSSS

Quality assessment for digital X-ray detector panels is an important issue, and evidently
impacts the quality of the resulting images. 
Detector panels being costly, 
decisions concerning replacement or refurbishing 
deserve careful justification.
% they can only 
% be replaced if this can be justified. 
Whether or not individual damaged pixels are actually 
problematic depends on where they are located and 
on how close they are to each other.
Thus spatial statistical analysis of dysfunctional pixels can be 
instrumental in assessing the relevance of damage.
An overview of common types of spatial patterns of dysfunctional pixels can be found in \cite{brettschneider2014crism}. 

The \DetectorCheckerWebApp is freely accessible via the url \footnote{
	Problems with running the web app are frequently resolved as follows: check (a) web browser is full upgraded; (b) adblocker plugins are disabled or not interfering with operation of web app; (c) Javascript blockers (e.g.~NoScript) are disabled or not interfering with operation of web app.}
:

\begin{mdframed}
	\bigskip
	\centerline{
		{\large
			\url{https://detectorchecker.azurewebsites.net}
		}
	}
	\medskip
\end{mdframed}

The web app allows users to assess digital X-ray detector panel quality via spatial analysis of dysfunctional pixels.
It is built using the \Rsoftware webtool \emph{shiny}, based on an open source \Rsoftware-package of the same name (\url{https://shiny.rstudio.com/}),
and is connected to an open archive of detector data contributed by the user community:
it is envisaged that users will be able to choose to create an account to upload their data to this archive,
to facilitate more extensive analysis of data trends concerning detector damage.

The quality assessment provided by \DetectorChecker can be used to support decision making 
about refurbishment and purchase of digital X-ray detectors. 
It can also be used to monitor the state of the detector over time and to link this to particular types of usage.
It is hoped that \DetectorChecker will assist in highlighting the occurrence of particular damage patterns, thus guiding
users in assessing likely reasons for such damage.

At the beginning of a session, 
\DetectorChecker allows users either (a) to select one of the common detector layouts,
or (b) to enter parameters defining a custom layout. 
Users can then upload data in the form of \texttt{xml} or \texttt{tif} files;
\DetectorChecker will then identify dysfunctional pixels.
Users can then instruct
\DetectorChecker to create numerical and graphical summaries 
about the spatial distribution of the dysfunctional pixels,
either on the pixel level or on the \emph{event level}
(automatically identifying and working with clusters of damage). The event level is likely to be more
closely linked to the reasons for damage, hence may provide a clearer picture 
of what is going on.
Finally, users can request that severely damaged regions of the detector are masked so as 
to exclude them from the analysis. This enables the effect of one specific damaged region 
to be decoupled from an assessment of the remainder of the detector.

\DetectorChecker has been developed as follows:
JAB wrote the original \Rsoftware-code and contributed to the design of the web app, 
TL converted the code into an \Rsoftware-package
\DetectorChecker,
and designed and built the 
\DetectorCheckerWebApp, 
which was further developed by OTG, while WSK initiated, supervised and tested the project at all stages. 
 
The authors thank The Alan Turing Institute for providing funding for this project and to Director of Research Engineering Martin O'Reilly. They are also grateful to Jay Warnett (at Warwick Manufacturing Group) and Andrew Ramsay (during his time at Nikon Metrology, UK) for data from Perkin Elmer detectors and for guidance on X-ray machines and damages. They also wish to thank Nicola Tartoni and Ian Horswell (at Diamond Lightsource, UK) for guidance on detector types and data from Pilatus and Excalibur detectors. Further thanks for helpful discussions and feedback are given
to Martin Turner and Tristan Lowe (University Manchester) and Martin O'Reilly (The Turing Institute).
\DetectorChecker and \DetectorCheckerWebApp build on previous work supported by EPSRC grant EP/K031066/1, and were funded by Turing seed fund projects awarded by the Alan Turing Institute under EPSRC grant EP/N510129/1.



%SSSSSSSSSSSSSSSSSSSSSSSSSSSSSSSSSSSSSSSSSSSSSSSSSSSSSSSSSSSSSSSSSSSS
\section{Workflow}\label{flow}
%SSSSSSSSSSSSSSSSSSSSSSSSSSSSSSSSSSSSSSSSSSSSSSSSSSSSSSSSSSSSSSSSSSSS

The interactive web app is structured around tabs that allow the user to choose actions from the left-hand panel and to view numerical and graphical results on the right-hand panel. Tooltips (in pop-up boxes) are available.
Outputs can be saved using standard mouse click functions from the operating system. 
Figure \ref{fig_flowchart} presents web app functionalities and possible work-flows
graphically.

\begin{figure}[htbp]
	\begin{center}
		\includegraphics[width=14.5cm]{flowchartDCshort.pdf}
		\vspace{-7mm}
		\caption
		{
			{\bfseries 
				Possible workflows using \DetectorChecker, involving
				the \emph{Layout}, \emph{Damage} and
				\emph{Model fitting} tabs.}
		}
		\label{fig_flowchart}
	\end{center}
\end{figure}
The basic work flow consists of eight steps, some of which can be run multiple times with different options. 
An overview of the roles of these steps is given in the box below and more details are explained in the remaining sections.

\vfill\newpage

\begin{mdframed}

\medskip

\begin{enumerate}

\item {\bf Select Layout:} 
Choose from a list of common detector layouts using a drop down menu. Alternatively, upload a plain text file with your own layout specifications (see Sections \ref{layout} and \ref{examples}).

\item {\bf Visualisation:} 
The default visualisation is a graphical representation of the layout.
Alternatively, pixel-wise functions related to the layout can be visualised. 
This includes measuring the distances to the centre, the corners or the edges of the panel or the edges of the sub-panels. 
These spatial functions are provided as covariates in Step 7.
% Both the layout itself and the spatial functions can be plotted.

\item {\bf Import File:} 
Upload a file with information on dysfunctional pixel locations. 
This can be an \texttt{xml} file containing the coordinates of the damaged pixels, or a \texttt{tif} or \texttt{hdf} file containing an array of pixel intensities corresponding to the mask describing the dysfunctional pixels. Examples for common detector layouts and for two user-defined layouts are included in the web app GitHub archive. Details are explained in Section \ref{dysfct} and \ref{examples}.
 
\item {\bf Choose Level:} 
Damage can be assessed at two different levels (see Section \ref{levels}). If the \emph{Pixels} option is selected, each damaged pixel is treated as an equally relevant occurrence. If \emph{Events} is selected then combinations of adjacent pixels form the basis of analysis, according to user choice of combination rules. 

\item {\bf Choose Analysis:} 
A range of spatial statistics tools to visualise the spatial patterns of damage is available (see Section \ref{spatial}). These include fitting a density, counts of dysfunctional pixels stratified by sub-panel, arrow between nearest neighbours and rose plots of their angles. 
Furthermore, the degree of complete spatial randomness can be explored by
plotting $K$-, $F$- and $G$-functions with both homogenous and inhomogeneous intensity.

The user can repeat this step running through different choices. Mouse-clicks on individual sub-panels produce individual plots showing only the sub-panel in question: double-clicking displays the corresponding plot or report for the sub-panel in question.

\item {\bf Send Data:} 
Optional data uploading using email address as unique identifier. This can be used to share data with the developers and potentially to share with user communities.

\item {\bf Modelling Damage Intensity:} Gives statistical guidance in identification of factors that have an influence on the decay of pixels. Choose a covariate related to detector layout for use in a linear model for pixel damage, and fit the model.


%\item  {\bf Mask Panel:} 
%Areas of intense damage may dominate the results of the analysis. They can be removed via a mask
%to allow the analysis of the remaining part of the detector panel (see section \ref{mask}).
% choose level
\end{enumerate}

\medskip

\end{mdframed}

\vfill\newpage

\medskip

%SSSSSSSSSSSSSSSSSSSSSSSSSSSSSSSSSSSSSSSSSSSSSSSSSSSSSSSSSSSSSSSSSSSS
\section{Detector Layout}\label{layout}
%SSSSSSSSSSSSSSSSSSSSSSSSSSSSSSSSSSSSSSSSSSSSSSSSSSSSSSSSSSSSSSSSSSSS

Preset layouts can be chosen corresponding to common detector panels. 
The web app will automatically display the chosen layout in the sub-tab \emph{Layout Analysis.} The sub-tab \emph{Summary} lists the parameters of the layout. 
\\
$\medblacktriangleright$ In Step 1 select from the drop-down menu offering the choice of Pilatus, several types of Perkin Elmer (full, cropped, refurbished), Excalibur and user-defined.

The last option refers to layouts specified by the user. It is assumed to be composed of \emph{sub-panels} arranged in a rectangular grid. They are also referred to as \emph{modules} or \emph{read out groups (ROG)}. Sub-panels can be seamless (as in the Perkin Elmer design; for details see \cite{manualXRD1621}), or include gaps between rows and/or columns. The dimensions of the sub-panels and the sizes of the gaps can vary, so long as the resulting layout still forms a grid. In other words, the gaps between adjacent columns of sub-panels need to be the same in all rows, and the gaps between adjacent rows need to be the same in all columns. Figure \ref{fig_layout_nopar} shows an example of such a layout. Appendix A gives two examples, a single-panel layout with photographic aspect ratio and an irregular layout with varying sub-panel sizes and varying gap sizes (see Section \ref{examples} on how to download them).
 
\begin{figure}[htbp]
\begin{center}
\includegraphics[width=5cm, angle=90]{layoutDrawingNoPar.pdf}
\caption
{
{\bfseries 
General layout with sub-panels and gaps.}
}
\label{fig_layout_nopar}
\end{center}
\end{figure}

To specify your own layout:
\\
$\medblacktriangleright$ In Step 1 choose \emph{user-defined}.  A ``Browse" button appears. Click on the button
to locate and upload a plain text file containing a list of parameters specifying the layout. Note that the plain text file will have to have the extension ``\texttt{.txt}''. (Users who use MS Word to construct the file should export the result as a ``\texttt{.txt}'' file.) 



%SSSSSSSSSSSSSSSSSSSSSSSSSSSSSSSSSSSSSSSSSSSSSSSSSSSSSSSSSSSSSSSSSSSS
\section{Dysfunctional pixels}\label{dysfct}
%SSSSSSSSSSSSSSSSSSSSSSSSSSSSSSSSSSSSSSSSSSSSSSSSSSSSSSSSSSSSSSSSSSSS

Dysfunctional pixels are referred to by many names including \emph{bad, damaged, dead, erratic, stuck, hot, defective, broken and underperforming.} Maintenance protocols may include the creation of so-called \emph{bad pixel maps} containing their locations. Pixels flagged as damaged in such maps are identified by a combination of criteria based on signal intensities, noise levels, uniformity and lag (see e.g.~Section 5.2 in \cite{manualXRD1621}). 
Common formats for bad pixel maps are \texttt{xml} for coordinates or \texttt{tif} or \texttt{hdf} for masks. 
\DetectorChecker can process all of these formats. To access examples files of dysfunctional pixel locations for both the built-in and the user-defined layouts see Section \ref{examples}.

To specify your dysfunctional pixel file:
\\
$\medblacktriangleright$ Click in the ``Browse" button 
to upload (from your computer) a file specifying bad pixels for your chosen layout, to be found in a sub-directory of \texttt{inst/extdata} named according to your chosen layout. Note that such files will have to have one of the extensions ``\texttt{.xml}'' (actually a text file listing bad pixel coordinates), ``\text{.tif}'' (an image derived from a test run) or ``\text{.hdf}'' (an alternative image specification found in the case of Excalibur: note that in this case you will need to shift-click to select and load all six files ``\texttt{pixelmask.fem*.hdf}`` together).

After uploading the file, \DetectorChecker automatically produces an image indicating the dysfunctional pixel locations 
by open squares. The characters are not plotted to scale but are magnified to ensure visibility of isolated damaged pixels.

We distinguish the types of spatial arrangements of dysfunctional pixels defined in \cite{brettschneider2014crism}:
singletons, doublets, triplets, larger clusters, upper/lower vertical lines, and right/left horizontal lines.
%\begin{itemize}
%\item[1.] singletons
%\item[2.] doublets
%\item[3.] triplets
%\item[4.] larger clusters
%\item[5./6.] upper/lower vertical lines
%\item[7./8.] right/left horizontal lines
%\end{itemize} 

An example for a highly damaged detector panel is shown in Figure \ref{fig_damagedPixels}.
It contains singleton dysfunctional pixels scattered all over the panel, 
lines of dysfunctional pixels intersecting with the horizontal midline indicating the two rows of sub-panels, a small cluster of dysfunctional pixels in the top right corner, and a high density region in the bottom right corner. 

\begin{figure}[htbp]
\begin{center}
\includegraphics[width=10cm]{damagedPixels.png}
\caption
{
{\bfseries Representation of detector panel with dysfunctional pixels.}
%B$\_$0 showing singletons, lines and corner damage. F$\_$0 showing also a high density region of damaged pixels.}.
}
\label{fig_damagedPixels}
\end{center}
\end{figure}


%SSSSSSSSSSSSSSSSSSSSSSSSSSSSSSSSSSSSSSSSSSSSSSSSSSSSSSSSSSSSSSSSSSSS
\section{Examples for user-defined layout files and dysfunctional pixel files}\label{examples}
%SSSSSSSSSSSSSSSSSSSSSSSSSSSSSSSSSSSSSSSSSSSSSSSSSSSSSSSSSSSSSSSSSSSS

For each of the built-in detector layouts, sample files with dysfunctional pixel locations are provided.
There is a dysfunctional pixel mask in \texttt{tif} format for the Pilatus detector layout. There are two sample  \texttt{xml} files giving two different examples of dysfunctional pixel coordinates for each of the Perkin Elmer layouts. 
There are also five \texttt{hdf} format which together give an example for the locations of disfunctional pixels in an Excalibur detector.

To illustrate the use of user-defined layouts two example layouts ``\texttt{.txt}'' files are included (see Appendix A). 
The first example is ``\texttt{layout\_par\_photographic.txt}'',  a single-panel photographic ratio layout. Three example for dysfunctional pixel locations are provided, a mask in ``\texttt{.tif}'' format and two coordinate lists in ``\texttt{.xml}'' format. (The latter ones have been simulated from homogeneous and inhomogeneous planar Poisson point patterns.)
The second example is ``\texttt{layout\_par\_irregular.txt}'', an irregular layout with varying sub-panel sizes and varying gap sizes. 

All examples are listed in Figure \ref{fig_examplesDirectory} in Appendix B and can be found in sub-directories of
\href{https://github.com/alan-turing-institute/DetectorChecker/tree/master/inst/extdata}{DetectorChecker/tree/master/inst/extdata} 
of the
\href{https://github.com/alan-turing-institute/DetectorChecker}{package GitHub repository}
(pointed to by \emph{Examples} in the \emph{Help} tab of the web app).

To use these examples, first download them from the GitHub repository onto your computer and then upload them into the web app in steps 1 and 3 as described in Sections \ref{layout} and \ref{dysfct}. If you need several of them the most efficient way may be to download the whole  \href{https://github.com/alan-turing-institute/DetectorChecker}{\DetectorChecker package}
as a ZIP file by pressing on the green ``Clone or download" button. Alternatively, you can download an individual file by clicking through subfolders until you find it, then clicking
on the file. 
For files in \texttt{tif} and \texttt{hdf} format, 
a ``Download" button will appear and should also be clicked on. 
For the examples in \texttt{xml} format, GitHub will offer a ``Raw" button, 
which displays the file on the browser. 
You can now manually transfer this to your computer into a file with extension ``\texttt{xml}" either by using copy and paste or by right-clicking and choosing \emph{page source.} 
Advice on problems with downloading from GitHub can be found online, 
e.g.~in the
\href{https://youtu.be/GIJdfuAoqFI}{``How to download files and Gists from GitHub" video by YouTuber Ugotsta}.


%SSSSSSSSSSSSSSSSSSSSSSSSSSSSSSSSSSSSSSSSSSSSSSSSSSSSSSSSSSSSSSSSSSSS
\section{Levels of Assessment}\label{levels}
%SSSSSSSSSSSSSSSSSSSSSSSSSSSSSSSSSSSSSSSSSSSSSSSSSSSSSSSSSSSSSSSSSSSS

While the natural entity of damage are pixels, this may
be neither the best way
to detect the reasons for damage nor 
the best way to rank their importance. For this reason,
a notion of a higher level analysis was introduced in
\cite{brettschneider2017crism}, summarising 
groups of adjacent pixels as \emph{events}. 
Speaking in general terms, lines are represented by one of their ends, and small clusters by their centre. For more details about the motivation and the technical definitions related to pixel and event perspectives see \cite{brettschneider2017crism}.

\DetectorChecker permits use of either of the two alternative levels of quality assessment, \emph{pixels}
or \emph{events}, defaulting to \emph{pixels}.
To perform the alternative event based analysis:\\
$\medblacktriangleright$ In Step 4 choose \emph{Events}. The types listed in the previous section will pop up. Choose a subset of these types and click on the \emph{Plot} button.\\
$\medblacktriangleright$ In Step 5 choose one of the options for graphical analysis, as described in the previous section, and 
click on the \emph{Plot} button.



%SSSSSSSSSSSSSSSSSSSSSSSSSSSSSSSSSSSSSSSSSSSSSSSSSSSSSSSSSSSSSSSSSSSS
\section{Spatial analysis}\label{spatial}
%SSSSSSSSSSSSSSSSSSSSSSSSSSSSSSSSSSSSSSSSSSSSSSSSSSSSSSSSSSSSSSSSSSSS

Spatial arrangement of dysfunctional pixels
can provide important clues to the reasons for damage.
Step 5 of the \DetectorChecker work-flow 
offers a number of exploratory tool to investigate this.

To see approximate areas of concentrated damage:\\
$\medblacktriangleright$ Choose \emph{Density}.

To see numbers of dysfunctional pixels (or events) stratified by sub-panel:\\
$\medblacktriangleright$ Choose \emph{Counts}.

To visualise the directions between nearest dysfunctional pixels by arrows:
\\
$\medblacktriangleright$ Choose \emph{Arrows}.

To obtain the resulting distribution of all the angles corresponding to the arrows:\\ 
$\medblacktriangleright$ Choose \emph{Angles}.

To obtain plots of functions examining complete spatial randomness:\\ 
$\medblacktriangleright$ Choose \emph{K-func.}, 
\emph{F-func.}, or\emph{G-func.}.

To obtain plots of functions examining complete spatial randomness, while correcting for spatial inhomogeneity:\\ 
$\medblacktriangleright$ Choose \emph{Inhom.~K-func.}, 
\emph{~F-func.}, or \emph{~G-func.}.

Note that plots of \emph{K-func.}, 
\emph{F-func.}, and \emph{G-func.}
all use scales determined by the data
(the default option for the underlying \Rsoftware package).
Therefore scales are not harmonized between different plots.

Having made a choice, click on the \emph{Plot} button.
Recall that clicking on a sub-panel, followed by double-clicking,
produces a corresponding plot or report 
for the sub-panel in question.
In the case of a density map, the smoothing 
chosen for the sub-panel is different
from that chosen for the plot for the entire panel,
so it will not simply be a detail of the larger image.

For details on the statistical methods and interpretation underlying these tools see \cite{brettschneider2017crism}.


%SSSSSSSSSSSSSSSSSSSSSSSSSSSSSSSSSSSSSSSSSSSSSSSSSSSSSSSSSSSSSSSSSSSS
\section{Modelling}\label{model}
%SSSSSSSSSSSSSSSSSSSSSSSSSSSSSSSSSSSSSSSSSSSSSSSSSSSSSSSSSSSSSSSSSSSS

Step 2 provided the option to display some functions that assigned coordinate values to pixels: the distance to the edges, the corners or to the vertical or horizontal sub-panel boundaries, if applicable. These spatial functions can be used as covariates in a statistical model that aims to determine factors contributing to the decay of pixels. \\
$\medblacktriangleright$ To visualise the spatial covariates, in Step 2 choose a specific covariate
(options 2 to 7) and click on the \emph{Display plot} button.\\
$\medblacktriangleright$ In Step 7 choose a covariate and and click on the \emph{Fit model} button.

After a delay for computation,
the right hand side of the panel will display 
output from fitting a linear model (logistic regression) which will assess the size and significance value of the selected covariate. 


%SSSSSSSSSSSSSSSSSSSSSSSSSSSSSSSSSSSSSSSSSSSSSSSSSSSSSSSSSSSSSSSSSSSS
%\section{Masking}\label{mask}
%SSSSSSSSSSSSSSSSSSSSSSSSSSSSSSSSSSSSSSSSSSSSSSSSSSSSSSSSSSSSSSSSSSSS

%Explain density based thresholding and creating of a mask.
%For more details about the masking see \cite{} (second tech report).



%SSSSSSSSSSSSSSSSSSSSSSSSSSSSSSSSSSSSSSSSSSSSSSSSSSSSSSSSSSSSSSSSSSSS
%\section{Case Studies}\label{cases}
%SSSSSSSSSSSSSSSSSSSSSSSSSSSSSSSSSSSSSSSSSSSSSSSSSSSSSSSSSSSSSSSSSSSS


%BBBBBBBBBBBBBBBBBBBBBBBBBBBBBBBBBBBBBBBBBBBBBBBBBBBBBBBBBBBBBBBB
%\chapter{Bibliography}\nonumber
%\chapter*{Bibliography}\nonumber
\addcontentsline{toc}{chapter}{\numberline{}Bibliography}
%\section*{Bibliography}
\bibliographystyle{plain}
\bibliography{biblioManual}
%BBBBBBBBBBBBBBBBBBBBBBBBBBBBBBBBBBBBBBBBBBBBBBBBBBBBBBBBBBBBBBBB

\vfill\newpage
\section*{Appendix A}

To upload layouts in the web app as user-defined detector layouts the list of parameters below has to be stored in a text file. 

The first example is a single-panel detector with photographic aspect ratio built in one piece (no sub-panels).

\begin{verbatim}
	detector_width = 600
	detector_height = 400
	module_col_n = 1
	module_row_n = 1
	module_col_sizes = c(600)
	module_row_sizes = c(400)
	gap_col_sizes = c()
	gap_row_sizes = c()
	module_edges_col = NA
	module_edges_row = NA}
\end{verbatim}
	
The second example is a an irregular layout with varying sub-panel sizes and varying gap sizes.

\begin{verbatim}
detector_width = 1720
detector_height = 1060
module_col_n = 7
module_row_n = 5
module_col_sizes = c(100, 200, 300, 400, 300, 200, 100)
module_row_sizes = c(100, 200, 400, 200, 100)
gap_col_sizes = c(10,20,30,30,20,10)
gap_row_sizes = c(10,20,20,10)
module_edges_col = NA
module_edges_row = NA
\end{verbatim}

\newpage

\section*{Appendix B}
\begin{figure}[htbp]
\vspace{-3cm}
\begin{center}
\includegraphics[width=15cm]{examplesDirectory.pdf}
\vspace{-3cm}
\caption
{
{\bfseries Directory 
containing all example files for dysfunctional pixel locations and user-defined layouts.}
}
\label{fig_examplesDirectory}
\end{center}
\end{figure}

\end{document}
